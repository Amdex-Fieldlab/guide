\documentclass{article}
\usepackage{graphicx} % Required for inserting images
\usepackage{hyperref}

\title{AMdEX Fieldlab Documentation}
\author{Merrick Oost-Rosengren}
\date{June 2023}

\begin{document}

\maketitle

\section{Documentation}
The documentation of the code can be found at \url{https://amdex-fieldlab.github.io/documentation/index.html}

\section{Main Components}
\subsection{AmdexProperties}
The AmdexProperties object contains all the properties needed to run the fieldlab components, and can also be used to access properties needed for the specific implementation of the node.

The Amdex Properties can contain single properties or classes with their initialization files.
\subsubsection{Required Properties}
\begin{verbatim}
# Identity of this node
node.identity.id = nodeId
node.identity.member = memberId
node.identity.consortium = consortiumName
node.identity.type = Executor

# Passwords
passwords.class = eu.amdexfieldlab.core.util.security.PropertiesPasswords
passwords.path = passwords.properties

# Backbone settings
backbone.protocol.prototypekafka.class = eu.amdexfieldlab.prototype.kafka.KafkaBackbone
backbone.protocol.prototypekafka.signing.class = eu.amdexfieldlab.prototype.SimulatedSigning
backbone.protocol.prototypekafka.pollfrequency = 1000
backbone.protocol.prototypekafka.timeout = 1000
backbone.protocol.prototypekafka.propertiesfile = kafka.properties
backbone.protocol.prototypekafka.trace = true

# Certificate Store Settings
certificates.store.class = eu.amdexfieldlab.prototype.kafka.KafkaPublicCertificateStore
certificates.store.topic = Certificates

\end{verbatim}
\subsection{Backbone}
\subsection{CertificateManager}
\section{Helper Components}
\subsection{CertificateUtils}
\subsection{AsynchronousManager}
\subsection{Json}


\end{document}
